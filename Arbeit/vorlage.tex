%
%  Vorlage/Template fuer #EBT
%
%  Created by Prof. Dr. Detlef Kreuz on 2010-08-14.
%  Copyright (c) 2010 . All rights reserved.
%
\documentclass[12pt,toc=bib,toc=listof]{scrreprt}
\usepackage[ngerman]{babel} 
\usepackage[utf8]{inputenc}
\usepackage[T1]{fontenc}
\usepackage{lmodern}
\usepackage{setspace}

\usepackage{hyperref}
\hypersetup{
  ,colorlinks=true
  ,linkcolor=blue
  ,citecolor=blue
  ,filecolor=blue
  ,urlcolor=blue
  }

%%%%%%%%%%%%%%%%%%%%%%%%%%%%%%%%%%%%% % (fold)
% Vom Studierenden zu aendernde Werte
\newcommand{\ebttopic}{Docker}
\newcommand{\ebtstudentname}{Jerome Tagliaferri}
\newcommand{\ebtstudentid}{190530}
\urldef{\ebtstudentmail}\url{jtagliaf@stud.hs-heilbronn.de}
%
%%%%%%%%%%%%%%%%%%%%%%%%%%%%%%%%%%%%% % (end)

\usepackage{ifpdf}
\ifpdf
\usepackage[pdftex]{graphicx}
\else
\usepackage{graphicx}
\fi

\usepackage[headsepline]{scrpage2}
\pagestyle{scrheadings}
\clearscrheadfoot
\ihead{E-Business-Technologien: \ebttopic}
\ohead{\pagemark}
\renewcommand*{\chapterpagestyle}{scrheadings}
\renewcommand*{\chapterheadstartvskip}{}

\titlehead{\flushright
%\includegraphics{HHN_KOMPLETT_4C.jpg}
\includegraphics[scale=0.6]{HHN_ab_40_mm_4c_neg.png}
}
\subject{Fallstudie Entwicklungswerkzeuge (281761)}
\title{\ebttopic}
\author{\ebtstudentname\footnote{\ebtstudentid, \ebtstudentmail}}
%% Datum nie auf einen festen Wert setzen
\publishers{Eingereicht bei Paul Lajer}

%\pagestyle{headings}

\begin{document}
\pagenumbering{roman} 
\selectlanguage{ngerman}
\maketitle
\tableofcontents

\addchap{Abkürzungsverzeichnis} % (fold)
\label{sec:abkuerzungsverzeichnis}

\begin{description}
  \item[ABK:] ABKÜRZUNG 
\end{description}

% chapter abkuerzungsverzeichnis (end)

\listoffigures
\listoftables

\onehalfspacing
\newpage
\pagenumbering{arabic}

\chapter{Einleitung} % (fold)
\label{sec:einleitung}
In der Einleitung der Arbeit („Was ich sagen will“) wird der Untersuchungsgegenstand
kurz umrissen und gegenüber weiterer Thematiken abgegrenzt, ggf. können auch die
verwendeten Methoden und begriffliche Grundlagen knapp erläutert werden. Ziel ist
es, das Interesse des Lesers zu wecken. Dabei soll neben der Motivation zur Arbeit,
das Ziel der Arbeit aufgezeigt werden sowie ein kurzer Überblick über den Aufbau der
Arbeit sowie die methodische Vorgehensweise gegeben werden.

\section{Motivation} % (fold)
\label{sec:motivation}
- Warum ist dies ein relevantes Thema zur momentanen Zeit ?

% section motivation (end)

\section{Ziel der Arbeit} % (fold)
\label{sec:ziel_der_arbeit}
- Was soll den genau erarbeitet werden ?


% section ziel_der_arbeit (end)

\section{Vorgehensweise} % (fold)
\label{sec:vorgehensweise}
- wie wird die Arbeit nun aufgebaut sein, was sind die Schritte um Schlussendlich auf ein Ergebnis zu kommen

% section vorgehensweise (end)
% chapter einleitung (end)

\chapter{Grundlagen} % (fold)
\label{sec:grundlagen}
Die Gliederung des Hauptteils der Arbeit („Ich sage, was ich zu sagen habe“) wird
durch die Struktur des Themas vorgegeben. Hier liegt der Schwerpunkt jeder wissen-
schaftlichen Arbeit. Sinnvoll sind kurze Einleitungen am Beginn jedes Gliederungs-
punktes. Hierbei geht es darum, wichtige Details zu erläutern, welche man z.B. durch
grafische Darstellungen verdeutlichen kann. Beginnen Sie jede Überschrift immer mit
Großbuchstaben! Allgemein ist es sinnvoll, den Inhalt des Hauptteils vom Allgemeinen
zum Speziellen aufzubauen. Wichtig ist es, den roten Faden in der Arbeit beizubehal-
ten.

- Historische Einordnung von Docker / Grundtechnologie
- 

\section{Einordnung des Problems / „state of the art“}
\label{sec:Einordnung}



% chapter grundlagen (end)

%%% ggf. weitere Abschnitte

\chapter{Fazit} % (fold)
\label{sec:fazit}
Im Schlusswort / Fazit („Was ich gesagt habe und was daraus folgt“) kann ein kurzer
Rückblick auf das Thema erfolgen. Wenn das Thema dies zulässt, können auch Zu-
kunftsperspektiven aufgezeigt und höchst subjektive Bewertungen des Verfassers ein-
gebracht werden.

% chapter fazit (end)

\appendix
\begin{thebibliography}{99}
\raggedright
%%% Printquellen zuerst
%%% Beispiel
\bibitem{Th11} Manuel René Theisen:
 \emph{Wissenschaftliches Arbeiten: Technik -- Methodik -- Form};
 15.~Auflage; Vahlen; München 2011;
 ISBN 978-3-8006-3830-7

%%% Internetquellen: Beispiel
\bibitem{hhneb} \emph{Hochschule Heilbronn};
 \url{http://www.hs-heilbronn.de/};
 abgerufen am 14.08.2010
\end{thebibliography}
\end{document}

